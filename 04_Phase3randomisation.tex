\chapter{Phase III - randomisation}
c'est la partie que la prof n'aime pas.\\

Affectation des différents traitements aux unités expérimentales de manière aléatoire.\\

\begin{itemize}
    \item Comparaison non biaisée des traitements [selection bias]
    \item Si grand nombre d’unités expérimentales : variables concomitantes
distribuée également entre les groupes trt [confondant facteurs]
    \item Base de l’inférence statistique
\end{itemize}

Principe de randomisation :
\begin{itemize}
    \item chaque patient à la même chance de recevoir chacun des traitements étudiés
    \item l’assignement d’un patient à un groupe traitement n’influence pas le choix du traitement des autres patients
    \item on ne sait pas à l’avance quel traitement chacun des patients va recevoir
\end{itemize}
\section{Randomisation simple}
Assignation des patients au traitement expérimental (EXP) ou
au traitement contrôle (CRT) sans raffinement.\\

Si la taille d’échantillon est petite, la randomisation simple n’est pas sans risque et peut quand même conduire à un facteur confondant n’étant pas équitablement réparti dans les deux groupes.


\section{Randomisation stratifiée}
La randomisation simple est réalisée séparément dans chacune des strates de la population définie par le facteur confondant.\\

La randomisation simple est réalisée séparément dans chacune des strates de la population définie par le facteur confondant.\\

Si la taille d’échantillon est petite, la randomisation simple et la randomisation stratifiée peut conduire à un nombre différent de patients dans chacun des groupes traitements.

\section{Randomisation restreinte ou par bloc}
La randomisation restreinte ou par bloc permet de limiter le déséquilibre dans l’assignement des groupes traitements. On choisit au préalable une taille n de bloc et on s’assure qu’un équilibre traitement est atteint après que n patients soient entrés dans l’étude.\\

 
Viole les principes de base de la randomisation, et notamment il devient possible de deviner à l’avance le traitement pour certains patients.


\section{Randomisation par minimisation}

Dans la randomisation par minimisation, la probabilité pour chaque patient d’être assigné à un bras ou un autre est modifiée de façon à favoriser le bras qui va réduire l’imbalance.\\

 
Viole les principes de base de la randomisation, mais moins que la randomisation par bloc ; et notamment garde un caractère aléatoire pour tous les patients.


